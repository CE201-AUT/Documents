حاصل $F_1 \cdot F_2 $ برابر است با اجتماع هر دو:
\begin{latin}
	$F_1 \cdot F_2 = \prod M(0,3,4,5,6,7)$
\end{latin}


برای اثباط این قضیه درحالت کلی می‌توان گفت:

\begin{latin}
	\[
	F_1 = \prod (a_i + M_i); \quad F_2 = \prod (b_j + M_j); \quad F_1 F_2 = \prod (a_i + M_i) \prod (b_j + M_j)
	\]
	\[
	= (a_0 + M_0)(b_0 + M_0)(a_1 + M_1)(b_1 + M_1)(a_2 + M_2)(b_2 + M_2) \dots = (a_0 b_0 + M_0)(a_1 b_1 + M_1)(a_2 b_2 + M_2) \dots
	\]
	\[
	= \prod (a_i b_i + M_i)
	\]
\end{latin}

ماکسترم \( M_i \) در \( F_1 F_2 \) حضور دارد اگر و تنها اگر \( a_i b_i = 0 \) باشد، به عبارت دیگر، اگر \( a_i = 0 \) یا \( b_i = 0 \) باشد. ماکسترم \( M_i \) در \( F_1 \) حضور دارد اگر و تنها اگر \( a_i = 0 \) باشد. ماکسترم \( M_i \) در \( F_2 \) حضور دارد اگر و تنها اگر \( b_i = 0 \) باشد. بنابراین، ماکسترم \( M_i \) در \( F_1 F_2 \) حضور دارد اگر و تنها اگر در \( F_1 \) یا \( F_2 \) حضور داشته باشد. \\ \\


درمورد  $F_1 + F_2 $ چه می‌توان گفت؟

حاصل این عبارت برابر است با مشترکات دو ماکسترم. یعنی:

\begin{latin}
	$F_1+F_2=(0,4,6)$
\end{latin}


و برای اثباط آن می‌توان نوشت:


\begin{latin}
	\[
	F_1 = \sum_{i=0}^{2^n-1} (a_i m_i); \quad F_2 = \sum_{i=0}^{2^n-1} (b_i m_i); \quad F_1 + F_2 = \sum_{i=0}^{2^n-1} (a_i m_i) + \sum_{i=0}^{2^n-1} (b_i m_i)
	\]
	\[
	= a_0 m_0 + b_0 m_0 + a_1 m_1 + b_1 m_1 + a_2 m_2 + b_2 m_2 + \dots = (a_0 + b_0)m_0 + (a_1 + b_1)m_1 + (a_2 + b_2)m_2 + \dots
	\]
	\[
	= \sum_{i=0}^{2^n-1} (a_i + b_i)m_i
	\]
\end{latin}


مین‌ترم \( m_i \) در \( F_1 + F_2 \) حضور دارد اگر و تنها اگر \( a_i + b_i = 1 \) باشد، به عبارت دیگر، اگر \( a_i = 1 \) یا \( b_i = 1 \) باشد، بنابراین ماکسترم \( M_i \) در \( F_1 + F_2 \) حضور دارد اگر \( a_i = 0 \) و \( b_i = 0 \) باشد. بنابراین، ماکسترم \( M_i \) در \( F_1 + F_2 \) حضور دارد اگر و تنها اگر در هر دو \( F_1 \) و \( F_2 \) حضور داشته باشد.

