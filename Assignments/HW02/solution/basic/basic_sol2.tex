ابتدا جدول کارنو این دو تابع را رسم می‌کنیم:


\begin{latin}
	\begin{minipage}{0.48\textwidth}
		\centering
		\begin{karnaugh-map}[4][4][1][$x_2$][$x_1$][$x_4$][$x_3$](label=corner)
			\minterms{1,4,5,6,7,12,15}
			\indeterminants{3,11,13,14}
			\implicant{4}{14}
			\implicant{1}{7}
		\end{karnaugh-map}
		\caption{K-Map 1}
		$f=x_4+x_2x_3'$\\
		$PIs=x_4, x_2x_3',x_1x_2$\\
		$EPIs=x_4, x_2x_3'$\\
	\end{minipage}
	\hfill
	\begin{minipage}{0.48\textwidth}
		\centering
		\begin{karnaugh-map}[4][4][1][$B$][$A$][$D$][$C$](label=corner)
			\minterms{1,4,12,13,14}
			\indeterminants{3,5,7,11,15}
			\implicant{12}{14}
			\implicant{4}{13}
			\implicant{1}{7}
		\end{karnaugh-map}
		\caption{K-Map 2}
		$f=C'B+DA'+CD$\\
		$PIs=C'B,DA',CD, AB$\\
		$EPIs=C'B,DA',CD$\\
	\end{minipage}	
\end{latin}

