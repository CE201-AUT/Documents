می‌خواهیم با استفاده از چهار ورودی \((A, B, C, D)\) و یک نمایشگر هفت قطعه‌ای \lr{(7 Segment)} خروجی‌هایی مطابق جدول زیر مشاهده کنیم (به عبارت دیگر، می‌خواهیم یک \lr{Binary-to-7-Segment Decoder} طراحی کنیم).



\begin{latin}
	\begin{center}
		\begin{tabular}{|c|c|c|c|c|}
			\hline
			\textbf{A} & \textbf{B} & \textbf{C} & \textbf{D} & \textbf{Display} \\
			\hline\hline
			0 & 0 & 0 & 0 & 0 \\
			0 & 0 & 0 & 1 & 1 \\
			0 & 0 & 1 & 0 & 2 \\
			0 & 0 & 1 & 1 & A \\
			0 & 1 & 0 & 0 & 3 \\
			0 & 1 & 0 & 1 & 4 \\
			0 & 1 & 1 & 0 & P \\
			0 & 1 & 1 & 1 & 5 \\
			1 & 0 & 0 & 0 & 6 \\
			1 & 0 & 0 & 1 & C \\
			1 & 0 & 1 & 0 & 7 \\
			1 & 0 & 1 & 1 & 8 \\
			1 & 1 & 0 & 0 & U \\
			1 & 1 & 0 & 1 & 9 \\
			1 & 1 & 1 & 0 & E \\
			1 & 1 & 1 & 1 & F \\
			\hline
		\end{tabular}
	\end{center}
\end{latin}


\begin{enumerate}
	\item 
	جدول صحت برای نمایش وضعیت تمامی خروجی‌ها (وضعیت هر قطعه از \lr{7segment}) را بر اساس اسم‌گذاری‌های اسلاید ۱۸ از مجموعه اسلاید ۶ رسم کنید. راهنمایی: این جدول باید مشابه جدول اسلاید ۱۹ از همان مجموعه اسلاید باشد.
	
	\item 
	با استفاده از جدول کارنو توابع ساده شده هر یک از خروجی‌های هفت‌گانه را تعیین کنید (مشابه اسلاید ۲۰).
\end{enumerate}




